\documentclass{article}
\usepackage[utf8]{inputenc}
\usepackage[polish]{babel}
\usepackage[T1]{fontenc}
\usepackage{graphicx}
\usepackage{geometry}
\usepackage{hyperref}

\geometry{a4paper, margin=2.5cm}

\title{Temat B: Konfiguracja maszyny wirtualnej z systemem Ubuntu}
\author{Jan Kowalski (Twój Nr Indeksu)} % Tutaj wpisz swoje dane
\date{\today}

\begin{document}

\maketitle
\tableofcontents
\newpage

\section{Wstęp}
Celem niniejszej instrukcji jest przedstawienie procesu instalacji i konfiguracji maszyny wirtualnej (VirtualBox) oraz instalacji systemu operacyjnego Linux Ubuntu. Środowisko to posłuży do bezpiecznego testowania skryptów i nauki administracji systemem.

\section{Wymagania wstępne}
Przed przystąpieniem do pracy należy przygotować:
\begin{itemize}
    \item Obraz instalacyjny systemu Ubuntu (plik .iso).
    \item Zainstalowane oprogramowanie Oracle VM VirtualBox.
    \item Minimum 20 GB wolnego miejsca na dysku.
\end{itemize}

\section{Proces konfiguracji}

\subsection{Tworzenie nowej maszyny}
1. Otwórz program VirtualBox i kliknij ikonę \textbf{"New"}.
2. Wpisz nazwę maszyny (np. "Ubuntu-Lab") i wybierz typ systemu "Linux".
3. Przydziel pamięć RAM (zalecane minimum 4096 MB).

\subsection{Instalacja systemu}
Po uruchomieniu nowo utworzonej maszyny, wskaż pobrany wcześniej plik ISO. Rozpocznie się proces instalacji, który obejmuje wybór języka, układu klawiatury oraz utworzenie konta użytkownika.

\section{Weryfikacja instalacji}
Poniższy zrzut ekranu przedstawia poprawnie uruchomioną maszynę wirtualną z terminalem gotowym do pracy.

\begin{figure}[h]
    \centering
    % Upewnij się, że plik screen.png znajduje się w folderze screenshots
    \includegraphics[width=0.8\textwidth]{screenshots/screen.png}
    \caption{Pomyślnie uruchomiony system Ubuntu w VirtualBox}
\end{figure}

\section{Podsumowanie}
Maszyna wirtualna została poprawnie skonfigurowana i jest gotowa do realizacji zadań laboratoryjnych.

\end{document}